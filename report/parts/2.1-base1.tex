\section{Suma de enteros en base uno}

\subsection{MT Determinista de 1 cinta}

\subsubsection*{Diseño propuesto}
El algoritmo de resolución es el siguiente:

\begin{itemize}
    \item Ciclo:
    \begin{enumerate}[1.]
        \item Comenzar desde la izquierda y desplazarse hasta el extremo derecho.
        \item Si el último elemnto es un 1:
        \begin{enumerate}[1.]
            \item Borrarlo.
            \item Volver al extremo izquierdo.
            \item Añadir un 1 en la primera posición
        \end{enumerate}
        \item Si el último elemnto es un \$:
        \begin{enumerate}[1.]
            \item Borrarlo.
            \item Volver al extremo izquierdo.
            \item \textbf{Parar}.
        \end{enumerate}
    \end{enumerate}
\end{itemize}

El diseño de la máquina queda representado en la Figura \ref{fig:MT-1A}.

\begin{figure}[h]
    \includegraphics[width=\textwidth]{MT-1A.png}
    \caption{Implementación en JFLAP de MT-1A}
    \label{fig:MT-1A}
\end{figure}


\subsubsection*{Peor caso}
Dado que el formato de palabra que acepta la Máquina de Turing son dos conjuntos de 1 separados por el simbolo \$, el peor caso sería aquel en el que haya una mayor cantidad de 1 a la derecha del \$, ya que por cada 1 se requerirá una iteración para eliminarlo de la derecha y añadirlo en la izquierda, por lo que a más 1 más iteraciones serán necesarias.

\subsubsection*{Evaluación empírica}
Realizamos la evaluación empírica en el peor caso, tomando como $n$ el número de 1 a la derecha de \$, ya que se trataría del peor caso, y midiendo el número de pasos realizados para resolver el problema:

\begin{table}[h]
    \centering
    \begin{tabular}{lccc}
        Entrada & $n_d$ & Pasos \\
        \hline
        $\lambda$               & 0  & N/A \\
        \$1                     & 1  & 11  \\
        \$11                    & 2  & 22  \\
        \$111                   & 3  & 37  \\
        \$1111                  & 4  & 56  \\
        \$11111                 & 5  & 79  \\
        

    \end{tabular}
\end{table}


\subsubsection*{Coste computacional}
Para obtener el coste computacional del algoritmo, aplicaremos Diferencias Finitas, basándonos en los datos de la evaluación empírica:

\begin{table}[h]
    \centering
    \begin{tabular}{|l|c|c|c|c|c|c|c|}
        \hline
        $n$ & \textbf{1} & \textbf{2} & \textbf{3} & \textbf{4} & \textbf{5}\\ \hline
        $T(n)$ & \textbf{11} & \textbf{22} & \textbf{37} & \textbf{56} & \textbf{79}      \\ \hline
        \hline
        $A(n) = T(n) - T(n-2)$ &    & 11 & 15 & 19 & 23\\ \hline
        $B(n) = A(n) - A(n-2)$ &    &    &  4 &  4 &  4\\ \hline
        $C(n) = B(n) - B(n-2)$ &    &    &    &  0 &  0 \\ \hline
    \end{tabular}
\end{table}

Se obsrerva que las diferencias finitas segundas son constantes y nulas las terceras. Por podemos aproximar $T(n)$ con un polinomio de segundo orden, es decir, $T(n) = an^2 + bn + c$.\\

Para obtener los valores de $a$, $b$, y $c$, usaremos valores de $n$ y $T(n)$ obtenidos en la evaluación empírica:

\begin{subequations}
    \begin{gather}
        n = 1,\ T(0) = 11 \rightarrow a + b + c = 11 \\
        n = 2,\ T(0) = 15 \rightarrow 4a + 2b + c = 15 \\
        n = 3,\ T(0) = 19 \rightarrow 9a + 3b + c = 19
    \end{gather}
\end{subequations}

Resolviendo, $a=0$, $b=4$ y $b=7$, por lo que:

\begin{equation}
    T_{0A}(n) = 4n + 7
\end{equation}

\subsubsection*{Cota asintótica}
La cota asintótica queda defida como:

\begin{equation}
    O(n) = k \cdot g(n) \geq T(n)
    \label{eq:On}
\end{equation}

Donde $g(n)$ representa el orden de la cota superior, en este caso $g(n) = n$.\\

Al conocer $T_{0A}(n)$, si asumimos $n_0 = 10$, obtenemos $k \geq \frac{47}{10}$, por lo que la cota asintótica para esta máquina es:
\begin{equation}
    O_{0A}(n) = \frac{47}{10} n
\end{equation}


%----------
%Sección 2
%----------

\subsection{MT Determinista de 2 cintas}

\subsubsection*{Diseño propuesto}
El algoritmo de resolución es el siguiente:



\begin{itemize}
    \item Ciclo:
    \begin{enumerate}[1.]
        \item Comenzar desde la izquierda en ambas cintas e ir desplazandose hasta el extremo derecho.
        \item Si se encuentra un 1 en la cinta 1:
        \begin{enumerate}[1.]
            \item Borrarlo de la cinta 1.
            \item Añadir un 1 en la cinta 2.
        \end{enumerate}
        \item Si se encuentra un \$ en la cinta 1:
        \begin{enumerate}[1.]
            \item Borrarlo de la cinta 1.
            \item No añadir nada en la cinta 2 y no avanzar en dicha cinta
        \end{enumerate}
        \item Cuando se llega al final de la cinta 2, \textbf{Parar}.:
    \end{enumerate}
\end{itemize}
De esta forma la primera cinta quedaría vacía y la solución resultante de la suma quedaría en la segunda cinta\\

El diseño de la máquina queda representado en la Figura \ref{fig:MT-1A}.

\begin{figure}[h]
    \includegraphics[width=\textwidth]{MT-1B.png}
    \caption{Implementación en JFLAP de MT-1B}
    \label{fig:MT-1B}
\end{figure}
