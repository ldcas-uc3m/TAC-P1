\section{Suma de enteros en base uno}

%----------
% MT-1A
%----------

\subsection{MT Determinista de 1 cinta}

\subsubsection*{Diseño propuesto}
El algoritmo de resolución es el siguiente:

\begin{itemize}
    \item Ciclo:
    \begin{enumerate}[1.]
        \item Desplazarse hasta el extremo derecho.
        \item Si el elmento es un 1:
        \begin{enumerate}[1.]
            \item Borrarlo.
            \item Volver al extremo izquierdo.
            \item Añadir un 1 en la primera posición.
        \end{enumerate}
        \item Si el elemnto es un \$:
        \begin{enumerate}[1.]
            \item Borrarlo.
            \item Volver al extremo izquierdo.
            \item \textbf{Parar}.
        \end{enumerate}
    \end{enumerate}
\end{itemize}

\plotimplementation{MT-1A}


\subsubsection*{Peor caso}
Dado que el formato de palabra que acepta la Máquina de Turing son dos conjuntos de 1 separados por el simbolo \$, el peor caso sería aquel en el que haya una mayor cantidad de 1 a la derecha del \$, ya que por cada 1 se requerirá una iteración para eliminarlo de la derecha y añadirlo en la izquierda, por lo que a más 1 más iteraciones serán necesarias.

\subsubsection*{Evaluación empírica}
Realizamos la evaluación empírica en el peor caso, tomando como $n$ el número de 1 a la derecha de \$, ya que se trataría del peor caso, y midiendo el número de pasos realizados para resolver el problema\footnote{Los datos se pueden encontrar en \texttt{data/MT-1A.csv}.}:

\begin{table}[h]
    \centering
    \begin{tabular}{lcc}
        Entrada                & $n$ & Pasos \\
        \hline
        $\lambda$               & 0  & N/A \\
        \$1                     & 1  & 11  \\
        \$11                    & 2  & 22  \\
        \$111                   & 3  & 37  \\
        \$1111                  & 4  & 56  \\
        \$11111                 & 5  & 79  \\
    \end{tabular}
    \caption{Tamaño de palabra y número de pasos realizados para MT-1A}
\end{table}


\subsubsection*{Coste computacional}
Para obtener el coste computacional del algoritmo, aplicaremos Diferencias Finitas\footref{dfref}, basándonos en los datos de la evaluación empírica:

\begin{table}[h]
    \centering
    \begin{tabular}{|l|c|c|c|c|c|}
        \hline
        $n$ & \textbf{1} & \textbf{2} & \textbf{3} & \textbf{4} & \textbf{5}\\ \hline
        $T(n)$ & \textbf{11} & \textbf{22} & \textbf{37} & \textbf{56} & \textbf{79}      \\ \hline
        \hline
        $A(n) = T(n) - T(n-2)$ &    & 11 & 15 & 19 & 23\\ \hline
        $B(n) = A(n) - A(n-2)$ &    &    &  4 &  4 &  4\\ \hline
        $C(n) = B(n) - B(n-2)$ &    &    &    &  0 &  0 \\ \hline
    \end{tabular}
    \caption{Aplicación de Diferencias Finitas a MT-1A}
\end{table}

Se obsrerva que las diferencias finitas segundas son constantes y nulas las terceras, por lo que podemos aproximar $T(n)$ con un polinomio de segundo orden, es decir, $T(n) = an^2 + bn + c$.\medskip

Para obtener los valores de $a$, $b$, y $c$, usaremos valores de $n$ y $T(n)$ obtenidos en la evaluación empírica:

\begin{subequations}
    \begin{gather*}
        n = 1,\ T(1) = 11 \rightarrow a + b + c = 11 \\
        n = 2,\ T(2) = 22 \rightarrow 4a + 2b + c = 22 \\
        n = 3,\ T(3) = 37 \rightarrow 9a + 3b + c = 37
    \end{gather*}
\end{subequations}

Resolviendo, $a=2$, $b=5$ y $b=4$, por lo que:

\begin{equation}
    T_{\mathrm{1A}}(n) = 2n^2 + 5n + 4
\end{equation}

\subsubsection*{Cota asintótica}
Al conocer $T_{\mathrm{1A}}(n)$, podemos afirmar que $g(n) = n^2$. Si asumimos $n_0 = 10$, obtenemos $k \geq \frac{127}{50}$, por lo que la cota asintótica (definida en la ecuación \ref{eq:On}) para esta máquina es:
\begin{equation}
    O_{\mathrm{1A}}(n) = \frac{127}{50} n^2
\end{equation}


\plotcomplexity{MT-1A}



%----------
% MT-1B
%----------

\subsection{MT Determinista de 2 cintas}

\subsubsection*{Diseño propuesto}
El algoritmo de resolución es el siguiente:

\begin{itemize}
    \item Ciclo:
    \begin{enumerate}[1.]
        \item Comenzar desde la izquierda en ambas cintas e ir desplazandose hasta el extremo derecho.
        \item Si se encuentra un 1 en la cinta 0:
        \begin{enumerate}[1.]
            \item Borrarlo de la cinta 0.
            \item Añadir un 1 en la cinta 1.
        \end{enumerate}
        \item Si se encuentra un \$ en la cinta 0:
        \begin{enumerate}[1.]
            \item Borrarlo de la cinta 0.
            \item No añadir nada en la cinta 1 y retroceder una posisción.
            \item Avanzar hacia la derecha en la cinta 0 hasata llegar al final y mantenerse sin moverse en la cinta 1.
            \item Por cada 1 en la cinta 1 borrarlo y desplazarse hacia la izquierda en dicha cinta, añadir un 1 en la cinta 0 y avanzar a la derecha
            \item Cuando la cinta 1 queda vacia, \textbf{parar}.
        \end{enumerate}
    \end{enumerate}
\end{itemize}
De esta forma al finalizar la ejecución, la primera cinta contiene el resultado de la suma de ambos números y la segunda cinta queda vacía.\medskip


Se ha aplicado una optimización respecto al algortimo original, por el cual una vez se ha copiado el primer número en la segunda cinta, tras eliminarse el \$ de la primera cinta, se empiezan a pasar el número de la segunda cinta a la primera desde la izquierda, en vez de copiarlo desde la derecha ya que que en este caso concreto no influye el orden de los digitos.\medskip

\plotimplementation[0.5]{MT-1B}


\subsubsection*{Peor caso}
En este algoritmo, el pero caso será aquel en el que el primer número, esto es a al izquierda del \$, tenga un mayor numero de digitos. Esto se debe a que para cada digito a la izquierda del \$, este deberá copiarse a la sgunda cinta  para posteioriormente escribirse de nuevo en la primera


\subsubsection*{Evaluación empírica}
Realizamos la evaluación empírica en el peor caso, tomando como $n$ la longitud de la cadena de entrada.\footnote{Los datos se pueden encontrar en \texttt{data/MT-1A.csv}.}:

\begin{table}[h]
    \centering
    \begin{tabular}{lcc}
        Entrada                    & $n$ & Pasos \\
        \hline
        1\$                       & 3  &  4  \\
        11\$                      & 4  &  6  \\
        111\$                     & 5  &  8  \\
        1111\$                    & 6  &  10  \\
        11111\$                   & 7  &  12  \\
    \end{tabular}
    \caption{Tamaño de palabra y número de pasos realizados para MT-1B}
\end{table}


\subsubsection*{Coste computacional}
Para obtener el coste computacional del algoritmo, aplicaremos Diferencias Finitas\footref{dfref}, basándonos en los datos de la evaluación empírica:

\begin{table}[h]
    \centering
    \begin{tabular}{|l|c|c|c|c|c|c|c|}
        \hline
        $n$ & \textbf{3} & \textbf{4} & \textbf{5} & \textbf{6} & \textbf{7}\\ \hline
        $T(n)$ & \textbf{4} & \textbf{6} & \textbf{8} & \textbf{10} & \textbf{12}      \\ \hline
        \hline
        $A(n) = T(n) - T(n-2)$ &    & 2 & 2 & 2 & 2 \\ \hline
        $B(n) = A(n) - A(n-2)$ &    &   & 0 & 0 & 0 \\ \hline
    \end{tabular}
    \caption{Aplicación de Diferencias Finitas a MT-1B}
\end{table}

Se obsrerva que las diferencias finitas primeras son constantes y nulas las segundas, por lo que podemos aproximar $T(n)$ con un polinomio de primer orden, es decir, $T(n) = an + b$.\medskip

Para obtener los valores de $a$ y $b$, usaremos valores de $n$ y $T(n)$ obtenidos en la evaluación empírica:

\begin{subequations}
    \begin{gather*}
        n = 3,\ T(3) = 4 \rightarrow 3a + b = 4 \\
        n = 4,\ T(4) = 6 \rightarrow 4a + b = 6
    \end{gather*}
\end{subequations}

Resolviendo, $a = 2$ y $b = -2$, por lo que:

\begin{equation}
    T_{\mathrm{1B}}(n) = 2n - 2
\end{equation}

\subsubsection*{Cota asintótica}
La ecuación \ref{eq:On} no es aplicable para polinomios con términos negativos ($a,\ b,\ c < 0$), cuando $n_0$ es pequeño ($n \rightarrow 0$), por lo que para aproximar $O(n)$ podemos suprimir esos términos, es decir, $T'_{\mathrm{1B}}(n) = 2n$.

Al conocer $T_{\mathrm{1B}}(n)$, podemos afirmar que $g(n) = n$. Si asumimos $n_0 = 10$, usando $T'_{\mathrm{1B}}(n)$, obtenemos $k \geq 2$, por lo que la cota asintótica (definida en la ecuación \ref{eq:On}) para esta máquina es:
\begin{equation}
    O_{\mathrm{1A}}(n) = 2n
\end{equation}

\plotcomplexity{MT-1B}
