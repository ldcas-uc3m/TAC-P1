\part{Resumen}

El presente informe recoge los trabajos realizados para el desarrollo de la práctica de máquinas de Turing, incluyendo la elaboración de máquinas tanto deterministas como no deterministas, con una única cinta o varias. En cada apartado además de describir el funcionamiento de la máquina planteada, se analiza su rendimeinto en terminos de coste y se indican las pruebas llevadas a cabo.\\

El desarrollo de las máquinas de Turing se ha llevado a cabo empleado el progrma Jflap que permite a través de una interfaz gráfica, el trabajo de manera intuitiva con máquinas de Turing incluyendo el poder realizar pruebas con múltiples entredas, y llevar acabo ejecuciones detalladas paso a paso.\\

Para la realización de pruebas y poder obtener el coste de cada máquina en términos de pasos requieridos hasta finalizar su ejecución, se ha empleado en Python un simulador que permite dada una máquina y una plabra de entrada determinar el resulatdo de dicha ejecución y el número de pasos requeridos, siendo todos estos resultados son guardados para su posterior revisión. Para facilitar la interpretación de los resultados de una forma visual se ha empleaddo la librería de Pyhton matplotlib a fin de crear diferentes gráficas presentando los resultados obtenidos.\\

\newpage