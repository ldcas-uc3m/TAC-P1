\section{Palabras de estructura triplicada (I)}
Dado el alfabeto $\Sigma = \{a,b\}$ y la palabra $x \in \Sigma$, $\exists w \mid x = w \cdot w^{-1} \cdot w'$ ($w=w'$).

%----------
% MT-5A
%----------

\subsection{MT Multicinta Determinista}

\subsubsection*{Diseño propuesto}
El algoritmo de resolución es el siguiente:

\begin{enumerate}
    \item Leer \texttt{a} o \texttt{b} en la cinta 0 tres veces, y poner un \texttt{1} en la cinta 1.
    \item Si no es el final de la cinta, volver al paso anterior. En caso afirmativo, tendremos en la cinta 1 el tamaño de la palabra $w = n/3$, en base uno.
    \item Copiar $w'$ a la cinta 1. Por cada \texttt{1} en la cinta 1:
    \begin{enumerate}[1.]
        \item Copiar la letra de la cinta 0 a la cinta 1, y borrarla de la cinta 0, sobreescribiendo los \texttt{1} y moviendo a la izquierda ambas cintas.
    \end{enumerate}
    \item Comprobar $w^{-1}$. Por cada letra de la cinta 1:
    \begin{enumerate}[1.]
        \item Validar que las letras de ambas cintas son iguales, mover la cinta 0 a la izquierda, borrando las letras, y la cinta 1 a la derecha.
    \end{enumerate}
    \item Comprobar $w$. Por cada letra de la cinta 1:
    \begin{enumerate}[1.]
        \item Validar que las letras de ambas cintas son iguales y mover ambas cintas a la izquierda, borrando las letras.
    \end{enumerate}
    \item Poner un \texttt{1} en la cinta 0 y \textbf{parar}.
\end{enumerate}

\plotimplementation[0.9]{MT-5A}


\subsubsection*{Peor caso}
El peor caso es cuando es una palabra de la gramática ($w \cdot w^{-1} \cdot w$), puesto que tiene que comprobarla entera. La estructura de $w$ es irrelevante, puesto que las transiciones no dependen de cómo esté formada.

\subsubsection*{Evaluación empírica}
Realizamos la evaluación empírica en el peor caso, tomando como $n$ el tamaño de la palabra, y midiendo el número de pasos realizados para resolver el problema\footnote{Los datos se pueden encontrar en \texttt{data/MT-5A.csv}.}:

\begin{table}[h]
    \centering
    \begin{tabular}{lcc}
        Entrada & $n$ & Pasos \\
        \hline
        \texttt{aaa}                &  3  & 10 \\
        \texttt{aaaaaa}             &  6  & 16 \\
        \texttt{aaaaaaaaa}          &  9  & 22 \\
        \texttt{aaaaaaaaaaaa}       & 12  & 28 \\
    \end{tabular}
    \caption{Tamaño de palabra y número de pasos realizados para MT-5A}
\end{table}


\subsubsection*{Coste computacional}
Para obtener el coste computacional del algoritmo, aplicaremos Diferencias Finitas, basándonos en los datos de la evaluación empírica:

\begin{table}[H]
    \centering
    \begin{tabular}{|l|c|c|c|c|}
        \hline
        $n$    & \textbf{3}  & \textbf{6}  & \textbf{9}  & \textbf{12} \\ \hline
        $T(n)$ & \textbf{10} & \textbf{16} & \textbf{22} & \textbf{28} \\ \hline
        \hline
        $A(n) = T(n) - T(n-2)$ &   & 6 & 6 & 6 \\ \hline
        $B(n) = A(n) - A(n-2)$ &   &   & 0 & 0 \\ \hline
    \end{tabular}
    \label{tab:5A}
    \caption{Aplicación de Diferencias Finitas a MT-5A}
\end{table}

Al ser constantes las diferencias finitas primeras, y nulas las segundas, podemos aproximar $T(n)$ con un polinomio de primer orden, es decir, $T(n) = an + b$.\medskip

Para obtener los valores de $a$ y $b$, usaremos valores de $n$ y $T(n)$ obtenidos en la evaluación empírica:

\begin{subequations}
    \begin{gather*}
        n = 3,\ T(3) = 10 \rightarrow 3a + b = 10 \\
        n = 6,\ T(6) = 16 \rightarrow 6a + b = 16
    \end{gather*}
\end{subequations}

Resolviendo, $a=2$ y $b=4$, por lo que:

\begin{equation}
    T_{\mathrm{5A}}(n) = 2n + 4
\end{equation}


\subsubsection*{Cota asintótica}
Al conocer $T_{\mathrm{5A}}(n)$, podemos afirmar que $g(n) = n$. Si asumimos $n_0 = 10$, obtenemos $k \geq \frac{12}{5}$, por lo que la cota asintótica (definida en la ecuación \ref{eq:On}) para esta máquina es:
\begin{equation}
    O_{\mathrm{5A}}(n) = \frac{12}{5} n
\end{equation}

\plotcomplexity{MT-5A}


\subsubsection*{Inclusión de una tercera cinta}
El diseño se basa en el uso de una MT de dos cintas, dado que la inclusión de una tercera cinta no proporcionaría ningún beneficio.

En este problema no sólo hay que localizar $w$ (en este caso, dividimos la entrada entre tres y cogemos el último elemento), sino que hay que validar $w$ y $w^{-1}$.\\
Con dos cintas lo comprobamos secuencialmente, y la idea sería comprobar $w^{-1}$ en la cinta 0 mientras comprobamos $w$ en la cinta 2.
Podríamos copiar la entrada también a la tercera cinta, pero no habría forma de dejar el cabezal de la tercera cinta apuntando al principio de la entrada (al principio de $w$) sin realizar pasos extras, puesto que nuestro algoritmo realiza una pasada para calcular la longitud de $w$ y la pasada de vuelta para comprobar las palabras.



%----------
% MT-5B
%----------

\subsection{MT Multicinta No Determinista}

\subsubsection*{Diseño propuesto}
El algoritmo de resolución es el siguiente:

\begin{enumerate}
    \item Copiar $w$ a la cinta 1, dejando el cabezal 0 al principio de $w^{-1}$ y el cabezal 1 al final de $w$. Dado que es una máquina no determinista, asumiremos a cada paso que hemos copiado toda la palabra.
    \item Comprobar $w^{-1}$. Por cada letra de la cinta 1:
    \begin{enumerate}[1.]
        \item Validar que las letras de ambas cintas son iguales, mover la cinta 1 a la izquierda, borrando las letras, y la cinta 0 a la derecha.
    \end{enumerate}
    \item Comprobar $w$. Por cada letra de la cinta 1:
    \begin{enumerate}[1.]
        \item Validar que las letras de ambas cintas son iguales y mover ambas cintas a la izquierda, borrando las letras.
    \end{enumerate}
    \item Poner un \texttt{1} en la cinta 0 y \textbf{parar}.
\end{enumerate}

\plotimplementation[0.8]{MT-5B}


\subsubsection*{Peor caso}
Al igual que en la determinista, el peor caso es cuando es una palabra de la gramática ($w \cdot w^{-1} \cdot w'$), independientemente de la estructura.


\subsubsection*{Evaluación empírica}
Realizamos la evaluación empírica en el peor caso, tomando como $n$ el tamaño de la palabra, y midiendo el número de pasos realizados para resolver el problema\footnote{Los datos se pueden encontrar en \texttt{data/MT-5B.csv}.}:

\begin{table}[h]
    \centering
    \begin{tabular}{lcc}
        Entrada & $n$ & Pasos \\
        \hline
        \texttt{aaa}                &  3  & 5 \\
        \texttt{aaaaaa}             &  6  & 8 \\
        \texttt{aaaaaaaaa}          &  9  & 11 \\
        \texttt{aaaaaaaaaaaa}       & 12  & 14 \\
    \end{tabular}
    \caption{Tamaño de palabra y número de pasos realizados para MT-5B}
\end{table}


\subsubsection*{Coste computacional}
Para obtener el coste computacional del algoritmo, aplicaremos Diferencias Finitas, basándonos en los datos de la evaluación empírica:

\begin{table}[H]
    \centering
    \begin{tabular}{|l|c|c|c|c|}
        \hline
        $n$    & \textbf{3} & \textbf{6} & \textbf{9}  & \textbf{12} \\ \hline
        $T(n)$ & \textbf{5} & \textbf{8} & \textbf{11} & \textbf{14} \\ \hline
        \hline
        $A(n) = T(n) - T(n-2)$ &   & 3 & 3 & 3 \\ \hline
        $B(n) = A(n) - A(n-2)$ &   &   & 0 & 0 \\ \hline
    \end{tabular}
    \caption{Aplicación de Diferencias Finitas a MT-5B}
\end{table}

Al ser constantes las diferencias finitas primeras, y nulas las segundas, podemos aproximar $T(n)$ con un polinomio de primer orden, es decir, $T(n) = an + b$.\medskip

Para obtener los valores de $a$ y $b$, usaremos valores de $n$ y $T(n)$ obtenidos en la evaluación empírica:

\begin{subequations}
    \begin{gather*}
        n = 3,\ T(3) = 5 \rightarrow 3a + b = 5 \\
        n = 6,\ T(6) = 8 \rightarrow 6a + b = 8
    \end{gather*}
\end{subequations}

Resolviendo, $a=1$ y $b=2$, por lo que:

\begin{equation}
    T_{\mathrm{5B}}(n) = n + 2
\end{equation}


\subsubsection*{Cota asintótica}
Al conocer $T_{\mathrm{5B}}(n)$, podemos afirmar que $g(n) = n$. Si asumimos $n_0 = 10$, obtenemos $k \geq \frac{6}{5}$, por lo que la cota asintótica (definida en la ecuación \ref{eq:On}) para esta máquina es:
\begin{equation}
    O_{\mathrm{5B}}(n) = \frac{6}{5} n
\end{equation}

\plotcomplexity{MT-5B}


\subsubsection*{Inclusión de una tercera cinta}
El diseño se basa en el uso de una MT de dos cintas, dado que la inclusión de una tercera cinta no proporcionaría ningún beneficio, por un motivo similar al de la máquina determinista: hacemos una única pasada de la entrada.