\part{Apéndices}

\section{Instalación y ejecución}

\subsection*{Ejecutar en JFLAP\supercite{jflap}}
\begin{enumerate}
  \item Comprueba que tienes Java 8 o superior instalado, ejecutando \texttt{java --version}. Si no lo tienes, descarga e instala Java Runtime Enviroment con JDK 8+, tanto el Java SE\footurl{https://www.oracle.com/java/technologies/java-se-glance.html} como OpenJDK\footurl{https://openjdk.org/}.
  \item Descarga e instala JFLAP 7.1\footurl{https://www.jflap.org/}:
  \begin{verbatim}
wget https://www.jflap.org/jflaptmp/july27-18/JFLAP7.1.jar
  \end{verbatim}
  \item Ejecuta JFLAP con:
  \begin{verbatim}
java -jar JFLAP7.1.jar
  \end{verbatim}
  Puedes arreglar errores de escalado con el parámetro \texttt{-Dsun.java2d.uiScale}, e.g. \texttt{java -Dsun.java2d.uiScale=2.0 -jar JFLAP7.1.jar}
  \item Selecciona \texttt{File} $\rightarrow$ \texttt{Open...} y el archivo deseado (archivo \texttt{.jff} dentro de la carpeta \texttt{src/tm/}) para abrirlo. Ábrelo como una \texttt{Standard Turing Machine}.
  \item Usa \texttt{Input} $\rightarrow$ \texttt{Step...} para ejecutar paso a paso o \texttt{Input} $\rightarrow$ \texttt{Fast Run...} para ejecutar completo.
\end{enumerate}


\subsection*{Ejecución de los tests en Python}
Requiere Python\footurl{https://www.python.org/} 3.10+.
\begin{enumerate}
  \item Crea un \textit{virtual enviroment} en la carpeta \texttt{.venv/}:
  \begin{verbatim}
    python3 -m venv ./.venv
  \end{verbatim}
  \item Activa el entorno:
  \begin{itemize}
    \item Linux:
    \begin{verbatim}
source .venv/bin/activate
    \end{verbatim}
    \item Windows (PowerShell):
    \begin{verbatim}
& .\.venv\Scripts\Activate.ps1
    \end{verbatim}
  \end{itemize}
  \item Instala las dependencias:
  \begin{verbatim}
pip install -r requirements.txt
  \end{verbatim}
  \item Compila el Turing Machine Simulator\supercite{tmsimulator} con GNU/Make\footurl{https://www.gnu.org/software/make/}:
  \begin{verbatim}
cd turing-machine-simulator
make
cd ..
  \end{verbatim}
  Si estás en Windows, te recomendamos instalar WSL2\footurl{https://learn.microsoft.com/es-es/windows/wsl/install} y ejecutar `en Linux', o con GCC a través de MinGW-W64\footurl{https://www.mingw-w64.org/} (puedes encontrar binarios ya compilados en \url{https://github.com/niXman/mingw-builds-binaries}), compilando mediante el comando \texttt{gcc} que puedes encontrar dentro de \texttt{turing-machine-simulator/\allowbreak Makefile}.
  \item Ejecuta el \textit{script} con:
  \begin{verbatim}
python3 src/test.py
  \end{verbatim}
\end{enumerate}
