\section{Palabras de estructura triplicada (II)}
Dado el alfabeto $\Sigma = \{a,b\}$ y la palabra $x \in \Sigma$, $\exists w \mid x = w \cdot w' \cdot w''$ ($w=w'=w''$).

%----------
% MT2T-6A
%----------

\subsection{MT Multicinta Determinista}

\subsubsection*{Diseño propuesto}
El algoritmo de resolución es el siguiente:

\begin{enumerate}
    \item Leer \texttt{a} o \texttt{b} en la cinta 0 tres veces, y poner un \texttt{1} en la cinta 1.
    \item Si no es el final de la cinta, volver al paso anterior. En caso afirmativo, tendremos en la cinta 1 el tamaño de la palabra $w = n/3$, en base uno.
    \item Copiar $w''$ a la cinta 1. Por cada \texttt{1} en la cinta 1:
    \begin{enumerate}[1.]
        \item Copiar la letra de la cinta 0 a la cinta 1, y borrarla de la cinta 0, sobreescribiendo los \texttt{1} y moviendo a la izquierda ambas cintas.
    \end{enumerate}
    \item Poner el cabezal 1 al final de $w$.
    \item Comprobar $w'$. Por cada letra de la cinta 1:
    \begin{enumerate}[1.]
        \item Validar que las letras de ambas cintas son iguales, moviendo la cinta 0 a la izquierda, borrando las letras, y la cinta 1 a la derecha.
    \end{enumerate}
    \item Poner el cabezal 1 al final de $w$.
    \item Comprobar $w$. Por cada letra de la cinta 1:
    \begin{enumerate}[1.]
        \item Validar que las letras de ambas cintas son iguales moviendo la cinta 0 a la izquierda, y la cinta 1 a la derecha, borrando ambas cintas.
    \end{enumerate}
    \item Poner un \texttt{1} en la cinta 0 y \textbf{parar}.
\end{enumerate}

\plotimplementation[0.9]{MT2T-6A}


\subsubsection*{Peor caso}
El peor caso es cuando es una palabra de la gramática ($w \cdot w' \cdot w''$), puesto que tiene que comprobarla entera. La estructura de $w$ es irrelevante, puesto que las transiciones no dependen de cómo esté formada.


\subsubsection*{Evaluación empírica}
Realizamos la evaluación empírica en el peor caso, tomando como $n$ el tamaño de la palabra, y midiendo el número de pasos realizados para resolver el problema\footnote{Los datos se pueden encontrar en \texttt{data/MT2T-6A.csv}.}:

\begin{table}[h]
    \centering
    \begin{tabular}{lcc}
        Entrada & $n$ & Pasos \\
        \hline
        \texttt{aaa}                &  3  & 14 \\
        \texttt{aaaaaa}             &  6  & 22 \\
        \texttt{aaaaaaaaa}          &  9  & 30 \\
        \texttt{aaaaaaaaaaaa}       & 12  & 38 \\
    \end{tabular}
    \caption{Tamaño de palabra y número de pasos realizados para MT2T-6A}
\end{table}


\subsubsection*{Coste computacional}
Para obtener el coste computacional del algoritmo, aplicaremos Diferencias Finitas\footref{dfref}, basándonos en los datos de la evaluación empírica:

\begin{table}[H]
    \centering
    \begin{tabular}{|l|c|c|c|c|}
        \hline
        $n$    & \textbf{3}  & \textbf{6}  & \textbf{9}  & \textbf{12} \\ \hline
        $T(n)$ & \textbf{14} & \textbf{22} & \textbf{30} & \textbf{38} \\ \hline
        \hline
        $A(n) = T(n) - T(n-2)$ &   & 8 & 8 & 8 \\ \hline
        $B(n) = A(n) - A(n-2)$ &   &   & 0 & 0 \\ \hline
    \end{tabular}
    \caption{Aplicación de Diferencias Finitas a MT2T-6A}
\end{table}

Al ser constantes las diferencias finitas primeras, y nulas las segundas, podemos aproximar $T(n)$ con un polinomio de primer orden, es decir, $T(n) = an + b$.\medskip

Para obtener los valores de $a$ y $b$, usaremos valores de $n$ y $T(n)$ obtenidos en la evaluación empírica:

\begin{subequations}
    \begin{gather*}
        n = 3,\ T(3) = 14 \rightarrow 3a + b = 14 \\
        n = 6,\ T(6) = 22 \rightarrow 6a + b = 22
    \end{gather*}
\end{subequations}

Resolviendo, $a=\frac{8}{3}$ y $b=6$, por lo que:

\begin{equation}
    T_{\mathrm{2T-6A}}(n) = \frac{8}{3}n + 6
\end{equation}


\subsubsection*{Cota asintótica}
Al conocer $T_{\mathrm{6A}}(n)$, podemos afirmar que $g(n) = n$. Si asumimos $n_0 = 10$, obtenemos $k \geq \frac{49}{15}$, por lo que la cota asintótica (definida en la ecuación \ref{eq:On}) para esta máquina es:
\begin{equation}
    O_{\mathrm{2T-6A}}(n) = \frac{49}{15} n
\end{equation}

\plotcomplexity{MT2T-6A}


\subsubsection*{Inclusión de una tercera cinta} \label{sec:MT3T-6A}
El diseño se basa en el uso de una MT de dos cintas, pero la inclusión de una tercera cinta proporcionaría beneficios.

Se puede utilizar la tercera cinta para copiar $w$ en dos cintas en vez de una, y así evitándonos recorrer $w$ una de las veces para recolocar el cabezal.

\plotimplementation[0.8]{MT3T-6A}

Realizamos la evaluación empírica\footnote{Los datos se pueden encontrar en \texttt{data/MT3T-6A.csv}.}:

\begin{table}[h]
    \centering
    \begin{tabular}{lcc}
        Entrada & $n$ & Pasos \\
        \hline
        \texttt{aaa}                &  3  & 12 \\
        \texttt{aaaaaa}             &  6  & 19 \\
        \texttt{aaaaaaaaa}          &  9  & 26 \\
        \texttt{aaaaaaaaaaaa}       & 12  & 33 \\
    \end{tabular}
    \caption{Tamaño de palabra y número de pasos realizados para MT3T-6A}
\end{table}

Calculamos el coste computacional mediante Diferencias Finitas:

\begin{table}[H]
    \centering
    \begin{tabular}{|l|c|c|c|c|}
        \hline
        $n$    & \textbf{3}  & \textbf{6}  & \textbf{9}  & \textbf{12} \\ \hline
        $T(n)$ & \textbf{12} & \textbf{19} & \textbf{26} & \textbf{33} \\ \hline
        \hline
        $A(n) = T(n) - T(n-2)$ &   & 7 & 7 & 7 \\ \hline
        $B(n) = A(n) - A(n-2)$ &   &   & 0 & 0 \\ \hline
    \end{tabular}
    \caption{Aplicación de Diferencias Finitas a MT3T-6A}
\end{table}

Al ser constantes las diferencias finitas primeras, y nulas las segundas, podemos aproximar $T(n)$ con un polinomio de primer orden, es decir, $T(n) = an + b$.\medskip

Para obtener los valores de $a$ y $b$, usaremos valores de $n$ y $T(n)$ obtenidos en la evaluación empírica:

\begin{subequations}
    \begin{gather*}
        n = 3,\ T(3) = 12 \rightarrow 3a + b = 12 \\
        n = 6,\ T(6) = 19 \rightarrow 6a + b = 19
    \end{gather*}
\end{subequations}

Resolviendo, $a=\frac{7}{3}$ y $b=5$, por lo que:

\begin{equation}
    T_{\mathrm{3T-6A}}(n) = \frac{7}{3}n + 5
\end{equation}

Podemos comparar ambas máquinas gráficamente (con $n_0 = 3$) en la Figura \ref{fig:MT-6A}, observando que la de tres cintas tiene un coste menor:

\begin{figure}[H]
    \centering
    \includesvg[width=0.6\textwidth]{plot_MT-6A_comparison.svg}
    \caption{Comparación del coste computacional entre MT2T-6A y MT3T-6A}
    \label{fig:MT-6A}
\end{figure}



%----------
% MT-6B
%----------

\subsection{MT Multicinta No Determinista}

\subsubsection*{Diseño propuesto}
El algoritmo de resolución es el siguiente:

\begin{enumerate}
    \item Copiar $w$ a la cinta 1, dejando el cabezal 0 al principio de $w'$ y el cabezal 1 al final de $w$. Dado que es una máquina no determinista, asumiremos a cada paso que hemos copiado toda la palabra.
    \item Comprobar $w'$. Por cada letra de la cinta 1:
    \begin{enumerate}[1.]
        \item Validar que las letras de ambas cintas son iguales, moviendo ambas cintas a la derecha, borrando las letras de la cinta 0.
    \end{enumerate}
    \item Poner el cabezal 1 al principio de $w$.
    \item Comprobar $w''$. Por cada letra de la cinta 1:
    \begin{enumerate}[1.]
        \item Validar que las letras de ambas cintas son iguales, moviendo ambas cintas a la derecha, borrando las letras de la cinta 0.
    \end{enumerate}
    \item Poner un \texttt{1} en la cinta 0 y \textbf{parar}.
\end{enumerate}

\plotimplementation[0.9]{MT2T-6B}


\subsubsection*{Peor caso}
Al igual que en la determinista, el peor caso es cuando es una palabra de la gramática ($w \cdot w' \cdot w''$), independientemente de la estructura.


\subsubsection*{Evaluación empírica}
Realizamos la evaluación empírica en el peor caso, tomando como $n$ el tamaño de la palabra, y midiendo el número de pasos realizados para resolver el problema\footnote{Los datos se pueden encontrar en \texttt{data/MT2T-6B.csv}.}:

\begin{table}[H]
    \centering
    \begin{tabular}{|l|c|c|c|c|}
        \hline
        $n$    & \textbf{3} & \textbf{6}  & \textbf{9}  & \textbf{12} \\ \hline
        $T(n)$ & \textbf{9} & \textbf{14} & \textbf{19} & \textbf{24} \\ \hline
        \hline
        $A(n) = T(n) - T(n-2)$ &   & 5 & 5 & 5 \\ \hline
        $B(n) = A(n) - A(n-2)$ &   &   & 0 & 0 \\ \hline
    \end{tabular}
    \caption{Tamaño de palabra y número de pasos realizados para MT2T-6B}
\end{table}

Al ser constantes las diferencias finitas primeras, y nulas las segundas, podemos aproximar $T(n)$ con un polinomio de primer orden, es decir, $T(n) = an + b$.\medskip

Para obtener los valores de $a$ y $b$, usaremos valores de $n$ y $T(n)$ obtenidos en la evaluación empírica:

\begin{subequations}
    \begin{gather*}
        n = 3,\ T(3) = 9 \rightarrow 3a + b = 9 \\
        n = 6,\ T(6) = 14 \rightarrow 6a + b = 14
    \end{gather*}
\end{subequations}

Resolviendo, $a=\frac{5}{3}$ y $b=4$, por lo que:

\begin{equation}
    T_{\mathrm{2T-6B}}(n) = \frac{5}{3}n + 4
\end{equation}


\subsubsection*{Cota asintótica}
Al conocer $T_{\mathrm{2T-6B}}(n)$, podemos afirmar que $g(n) = n$. Si asumimos $n_0 = 10$, obtenemos $k \geq \frac{31}{15}$, por lo que la cota asintótica (definida en la ecuación \ref{eq:On}) para esta máquina es:
\begin{equation}
    O_{\mathrm{2T-6B}}(n) = \frac{31}{15} n
\end{equation}

\plotcomplexity{MT2T-6B}


\subsubsection*{Inclusión de una tercera cinta}
El diseño se basa en el uso de una MT de dos cintas, pero la inclusión de una tercera cinta proporcionaría beneficios, por los mismos motivos que en el apartado 

Se podría utilizar la tercera cinta para copiar $w$ a dos cintas en vez de una, y así evitándonos recorrer $w$ una de las veces para recolocar el cabezal.

\plotimplementation[0.9]{MT3T-6B}

Realizamos la evaluación empírica\footnote{Los datos se pueden encontrar en \texttt{data/MT3T-6B.csv}.}:

\begin{table}[h]
    \centering
    \begin{tabular}{lcc}
        Entrada & $n$ & Pasos \\
        \hline
        \texttt{aaa}                &  3  &  7 \\
        \texttt{aaaaaa}             &  6  & 11 \\
        \texttt{aaaaaaaaa}          &  9  & 15 \\
        \texttt{aaaaaaaaaaaa}       & 12  & 19 \\
    \end{tabular}
    \caption{Tamaño de palabra y número de pasos realizados para MT3T-6B}
\end{table}

Calculamos el coste computacional mediante Diferencias Finitas:

\begin{table}[H]
    \centering
    \begin{tabular}{|l|c|c|c|c|}
        \hline
        $n$    & \textbf{3} & \textbf{6}  & \textbf{9}  & \textbf{12} \\ \hline
        $T(n)$ & \textbf{7} & \textbf{11} & \textbf{15} & \textbf{19} \\ \hline
        \hline
        $A(n) = T(n) - T(n-2)$ &   & 4 & 4 & 4 \\ \hline
        $B(n) = A(n) - A(n-2)$ &   &   & 0 & 0 \\ \hline
    \end{tabular}
    \caption{Aplicación de Diferencias Finitas a MT3T-6B}
\end{table}

Al ser constantes las diferencias finitas primeras, y nulas las segundas, podemos aproximar $T(n)$ con un polinomio de primer orden, es decir, $T(n) = an + b$.\medskip

Para obtener los valores de $a$ y $b$, usaremos valores de $n$ y $T(n)$ obtenidos en la evaluación empírica:

\begin{subequations}
    \begin{gather*}
        n = 3,\ T(3) = 7 \rightarrow 3a + b = 7 \\
        n = 6,\ T(6) = 11 \rightarrow 6a + b = 11
    \end{gather*}
\end{subequations}

Resolviendo, $a=\frac{5}{3}$ y $b=3$, por lo que:

\begin{equation}
    T_{\mathrm{3T-6B}}(n) = \frac{4}{3}n + 3
\end{equation}


Podemos comparar ambas máquinas gráficamente (con $n_0 = 3$) en la Figura \ref{fig:MT-6B}, observando que la de tres cintas tiene un coste menor:

\begin{figure}[H]
    \centering
    \includesvg[width=0.6\textwidth]{plot_MT-6B_comparison.svg}
    \caption{Comparación del coste computacional entre MT2T-6N y MT3T-6N}
    \label{fig:MT-6B}
\end{figure}


\subsection*{Comparación}
Se pueden comparar todas las máquinas que resuelven este problema en la Figura \ref{fig:MT-6}:
\begin{figure}[H]
    \centering
    \includesvg[width=0.6\textwidth]{plot_MT-6_comparison.svg}
    \caption{Comparación del coste computacional entre MT2T-6N y MT3T-6N}
    \label{fig:MT-6}
\end{figure}