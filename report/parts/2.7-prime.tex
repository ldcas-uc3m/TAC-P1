\section{Diseño de una MTND para determinar \texttt{prime(n)}}

%----------
% MT-7A
%----------

\subsection{\texttt{noprime}}

\subsubsection*{Diseño propuesto}
El algoritmo de resolución es el siguiente:

\begin{enumerate}
    \item Copiar un \texttt{1} de la cinta 0 a la cinta 1, borrando la cinta 0. Esto se hace ya que el número $1$ es primo y divisor de todos los números, para así empezar a verificar divisores a partir del número $2$.
    \item Copiar el posible divisor $m$ a la cinta 1, borrando la cinta 0 y dejando el cabezal 1 al final de $m$. Dado que es una máquina no determinista, esto generará máquinas $\forall m \leq n$.
    \item Comprobar si $m$ es divisor de $n$, es decir, si $\exists p \mid m\cdot p = n$ ($p \in \mathbb{N}$). Esto se consigue recorriendo la cinta 1 ($m$) $p$ veces a la vez que recorremos la cinta 0 ($n$) y comprobando si se llega al final de la cinta 0 a la vez que se llega al final/principio de la cinta 1.\\
    Hasta que las dos cintas están vacías:
    \begin{enumerate}[1.]
        \item Avanzar la cinta 0, borrándola, y retroceder la cinta 1, mientras se lean \texttt{1}.
        \item Al llegar al principio de la cinta 1, avanzar ambas cintas, borrando la cinta 0, mientras se lean \texttt{1}.
        \item Al llegar al final de la cinta 1, volver a empezar.
    \end{enumerate}
    \item Limpiar la cinta 1, moviéndola a la izquierda o derecha dependiendo de si hemos acabado al principio o al final de $m$, y \textbf{parar}.
\end{enumerate}

\plotimplementation[1]{MT-7A}


\subsubsection*{Peor caso}


\subsubsection*{Evaluación empírica}
Realizamos la evaluación empírica en el peor caso, tomando como $n$ el tamaño de la palabra, y midiendo el número de pasos realizados para resolver el problema\footnote{Los datos se pueden encontrar en \texttt{data/MT-7A.csv}.}:

% \begin{table}[h]
%     \centering
%     \begin{tabular}{lcc}
%         Entrada & $n$ & Pasos \\
%         \hline
%         \texttt{aaa}                &  3  & 14 \\
%         \texttt{aaaaaa}             &  6  & 22 \\
%         \texttt{aaaaaaaaa}          &  9  & 30 \\
%         \texttt{aaaaaaaaaaaa}       & 12  & 38 \\
%     \end{tabular}
% \end{table}


\subsubsection*{Coste computacional}
Para obtener el coste computacional del algoritmo, aplicaremos Diferencias Finitas, basándonos en los datos de la evaluación empírica:

% \begin{table}[H]
%     \centering
%     \begin{tabular}{|l|c|c|c|c|}
%         \hline
%         $n$    & \textbf{3}  & \textbf{6}  & \textbf{9}  & \textbf{12} \\ \hline
%         $T(n)$ & \textbf{14} & \textbf{22} & \textbf{30} & \textbf{38} \\ \hline
%         \hline
%         $A(n) = T(n) - T(n-2)$ &   & 8 & 8 & 8 \\ \hline
%         $B(n) = A(n) - A(n-2)$ &   &   & 0 & 0 \\ \hline
%     \end{tabular}
% \end{table}

% Al ser constantes las diferencias finitas primeras, y nulas las segundas, podemos aproximar $T(n)$ con un polinomio de primer orden, es decir, $T(n) = an + b$.\\

% Para obtener los valores de $a$ y $b$, usaremos valores de $n$ y $T(n)$ obtenidos en la evaluación empírica:

% \begin{subequations}
%     \begin{gather}
%         n = 3,\ T(3) = 14 \rightarrow 3a + b = 14 \\
%         n = 6,\ T(6) = 22 \rightarrow 6a + b = 22
%     \end{gather}
% \end{subequations}

% Resolviendo, $a=\frac{8}{3}$ y $b=6$, por lo que:

% \begin{equation}
%     T_{\mathrm{2T-6A}}(n) = \frac{8}{3}n + 6
% \end{equation}


\subsubsection*{Cota asintótica}
% Al conocer $T_{\mathrm{7A}}(n)$, podemos afirmar que $g(n) = n$. Si asumimos $n_0 = 10$, obtenemos $k \geq \frac{49}{15}$, por lo que la cota asintótica (definida en la ecuación \ref{eq:On}) para esta máquina es:
% \begin{equation}
%     O_{\mathrm{7A}}(n) = \frac{49}{15} n
% \end{equation}

% \plotcomplexity{MT-7A}


