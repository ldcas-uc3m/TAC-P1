\part{Conclusiones}

\section{Desafios}
Al igual que en la práctica anterior, hemos aprovechado para aprender a automatizar pruebas con Python y a mostrarlas con la librería matplotlib\parencite{matplotlib}. También aprovechamos para empezar a trabajar con herramientas necesarias para realizar nuestro TFG, como son \LaTeX\parencite{latex} para realizar la memoria, y BibLaTeX\parencite{biblatex} para manejar las referencias.\medskip

A la hora de automatizar las pruebas, nos decantamos por crear las máquinas de Turing en JFLAP\parencite{jflap}, convertirlas mediante el \textit{script}\footnote{\texttt{src/jf2tm.py}} proporcionado (modificándolo para nuestros propósitos) al formato del simulador turingmachinesimulator\parencite{turingmachinesimulator}, y simularlas localmente a través de Turing Machine Simulator\parencite{tmsimulator}, un simulador open-source que lee el mismo formato que turingmachinesimulator. También se modificó el código de este simulador para que exportase los resultados en formato JSON\parencite{json} y así poder trabajar con ellos cómodamente en Python. Posteriormente se creó una \href{https://github.com/fcortes/turing-machine-simulator/pull/2}{\textit{pull request}} para contribuír el código.\medskip

Más allá de las dificultades técnicas, también tuvimos problemas al diseñar ciertas máquinas, como MT-2A y MT-7B, y al calcular ciertas cotas superiores mediante Diferencias Finitas.


\section{Conclusiones Generales}



\newpage