\part{Conclusiones}

\section{Desafios}
Al igual que en la práctica anterior, hemos aprovechado para aprender a automatizar pruebas con Python y a mostrarlas con la librería matplotlib \footcite[ver][]{matplotlib}. También aprovechamos para empezar a trabajar con herramientas necesarias para realizar nuestro TFG, como son \LaTeX \footcite[ver][]{latex} para realizar la memoria, y BibLaTeX \footcite[ver][]{biblatex} para manejar las referencias.\medskip

A la hora de automatizar las pruebas, nos decantamos por crear las máquinas de Turing en JFLAP \footcite[ver][]{jflap}, convertirlas mediante el \textit{script}\footnote{\texttt{src/jf2tm.py}} proporcionado (modificándolo para nuestros propósitos) al formato del simulador turingmachinesimulator \footcite[ver][]{turingmachinesimulator}, y simularlas localmente a través de Turing Machine Simulator \footcite[ver][]{tmsimulator}, un simulador open-source que lee el mismo formato que turingmachinesimulator. También se modificó el código de este simulador para que exportase los resultados en formato JSON y así poder trabajar con ellos cómodamente en Python. Posteriormente se creó una \textit{pull request}\footnote{\url{https://github.com/fcortes/turing-machine-simulator/pull/2}} para contribuír el código.\medskip

También nos topamos con otro problema técnico: JFLAP \footcite[ver][]{jflap} no exporta las imágenes en formato vectorial (SVG, etc.), sino en PNG y con un límite de calidad.\\
Encontramos entonces JFLAP2Ti\textit{k}Z \footcite[ver][]{jflap2tikz}, el cual convierte archivos \texttt{.jff} en \texttt{.tex}, para así luego exportar a PDF y reimportar a \LaTeX a través del comando \verb|\includegraphics|, pero nos topamos con el problema de que se perderían las notas de los estados, lo cual es algo que clarifica y ayuda bastante a la comprensión, por lo tanto no acabó siendo usado.\medbreak

Más allá de las dificultades técnicas, también tuvimos problemas al diseñar ciertas máquinas, como MT-2A y MT-7B, y al calcular ciertas cotas superiores mediante Diferencias Finitas.


\section{Conclusiones Generales}

En terminos generales consideramos que esta práctica ha resultado de gran utilidad para poder poner en práctica los conocimeinto teóricos obtenidos tanto de desarrollo de máquinas de Turing deterministas y no deterministas, permitiendonos familiarizarnos con una herramienta de gran utilidad para ello es como es JFLAP, así como la evaluación del rendimeinto de algoritmos en terminos de complejidad y coste computacional.
Estos conocimientos pueden resultar de gran ayuda de cara a futuros proyectos en los que se lleve acabo el desarrollo propio de algoritmos, a fin de poder optimizarlos y analizar posibles mejoras en su implementación.

\vfill

También queremos dejar una última figura, los costes computacionales de todas las máquinas en una pequeña ventana donde se puedan apreciar todas:
\begin{figure}[H]
  \centering
  \includesvg[scale=0.7]{plot_all_complexity.svg}
  \caption{Coste computacional de todas las MT}
\end{figure}
